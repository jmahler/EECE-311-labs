
%
% # INTRODUCTION:
%
% This write up is for Lab2: Verification of Equivalent Logic Functions
%
% Specifically this was used for the class Linear Circuits II (EECE 311)
% taught by Hede Ma [http://www.ecst.csuchico.edu/~hma/]
% during the Fall 2011 semester at CSU Chico [www.csuchico.edu].
% 
% ## LaTeX
%
% This file is written for LaTeX [http://www.latex-project.org/]
% which is used to process this file in to a completely formatted
% document.
%
% If you are unfamiliar with LaTeX it can seem daunting at first
% (as with anything new) but there are many benefits.
% Imagine writing a document in Word except without
% having to worry about the tedious things such as line breaks,
% indentation, table of contents, appendices, font styles,
% heading sizes, citations/references, page numbers.
% LaTeX lets you focus on the content without
% worrying about the tedious details.  It is also excellent for
% producing mathematical formulas.
% 
% If you are collaborating with someone else you can simply edit
% the sections and paragraphs in this file as needed.
%
% To process this file use a command such as 'rubber'.
%
%   bash$ rubber skel.tex
%   (output to skel.dvi)
%   bash% rubber --pdf skel.tex
%   (output to skel.pdf)
%
% # AUTHORS (of this template):
%
%   Jeremiah Mahler <jmmahler@gmail.com>
%   https://www.google.com/profiles/jmmahler#about 
%
% # COPYRIGHT:
%
%   Copyright (C)  2011 Jeremiah Mahler <jmmahler@gmail.com>.
%   Permission is granted to copy, distribute and/or modify this document
%   under the terms of the GNU Free Documentation License, Version 1.3
%   or any later version published by the Free Software Foundation;
%   with no Invariant Sections, no Front-Cover Texts, and no Back-Cover Texts.
%   A copy of the license is included in the file "fdl-1.3.txt".
%

%\documentclass[12pt]{article}
\documentclass{article}
%\documentclass[10pt]{article}

%\usepackage{mslapa}
\usepackage{hyperref}
\usepackage{amsmath}
\usepackage{graphicx}
\usepackage{ulem}
%\usepackage{vmargin}
\usepackage{tabularx}
\usepackage{sectsty}
\usepackage{pbox}
\usepackage{bigstrut}
\usepackage{enumerate}
\usepackage{listings}
\usepackage{parskip}  % space paragraphs but dont indent

%\usepackage{cleveref}

%\setpapersize{USletter}
\sectionfont{\normalsize}
\subsectionfont{\normalsize}

% configure \bigstrut size
% This configures spacing above and below rows in a tabularx.
%\renewcommand{\bigstrutjot}{6pt}
\renewcommand{\bigstrutjot}{2.0\jot}

%\setlength{\parindent}{0in}
%\setlength{\parindent}{1.5ex}
%\setlength{\parskip}{1ex plus 0.5ex minus 0.2ex}

\raggedright

\begin{document}

% If a figure is too long to fit in a figure on a single page
% it should got in its own section in the Appendix.

% {{{ Cover Page

\centerline{\bf EECE 311}
\centerline{\bf Fall 2011}
\centerline{\bf}
\centerline{\bf Lab Report \#1}
\centerline{\bf Using SPICE for Electric Circuit Analysis}
%\centerline{\bf Section 4}
%\centerline{\bf 9/13/2011} % date turned in
\centerline{\bf 9/6/2011}  % date lab performed

% signature area
\begin{center}
\begin{tabularx}{\textwidth}[b]{X l l}
Submitted by: & & \\
Signature & Printed Name & Date \\
\hline
\multicolumn{1}{|X|}{} & \multicolumn{1}{|l|}{\bigstrut \bf Jeremiah Mahler} & \multicolumn{1}{|l|}{\bf Sep 13, 2011} \\
\hline
%\multicolumn{1}{|X|}{} & \multicolumn{1}{|l|}{\bigstrut \bf Marvanee Johnson} & \multicolumn{1}{|l|}{\bf Sep 14, 2011} \\
%\hline
\end{tabularx}
\end{center}
% }}}

% {{{ Description/Objectives
\section{Description/Objectives}


The objective of this lab is to gain experience simulating and analyzing
circuits using SPICE simulations.

% }}}

% {{{ Procedure
\section{Procedure}

In general the procedure involves creating a SPICE definition of the circuit
and then analyzing the results to produce raw values or plots.
Shown in Figure \ref{fig:circuit} is the graphical representation of the
 circuit that will be used in this lab.

\begin{figure}[!hbtp]
\center
%\includegraphics[scale=1.0]{Lab1-311-circuit}
\includegraphics[scale=0.4]{spice/lab1-schematic}
\caption{Circuit used for SPICE simulation.
Nodes denoted by numbers with 0 as common.}
\label{fig:circuit}
\end{figure}

The SPICE definition of the circuit is given in Appendix \ref{sec:spicedef}.
To run the simulation there are various options depending on the platform
being used.
Under Windows, the closed source student version of the
Orcad \cite{ORCAD} simulator can be used.
It provides a graphical user interface its use is self explanatory.
Under Linux and Windows the open source Ngspice \cite{NGSPICE}
simulator can be used.
Shown in Figure \ref{fig:ngspicelinux} are the commands used under Linux
to produce all the text based data values and plots of the simulation.
Figures \ref{fig:spicecur}, \ref{fig:spicecurplot}, \ref{fig:spicevolt},
and \ref{fig:spicevoltplot} show the textual output produced from running
the SPICE simulation.

\begin{figure}[!htb]
{\footnotesize
\begin{verbatim}
	shell$ # ngspice -b <name of spice file>
	shell$ ngspice -b lab1.cir
	(output sent to screen)
	shell$ ngspice -b lab1.cir > lab1.out
	(output save to file lab1.out)
\end{verbatim}
}
\caption{Shell commands for running the SPICE simulation in batch mode (-b)
using Ngspice under Linux.}
\label{fig:ngspicelinux}
\end{figure}

% {{{ fig:spicecur
\begin{figure}
{\footnotesize
\begin{verbatim}
                         eece 311 laboratory project #1
                         DC transfer characteristic  Sun Sep 11 11:14:25  2011
--------------------------------------------------------------------------------
Index   v-sweep         vs1#branch      vsens#branch    
--------------------------------------------------------------------------------
0    0.000000e+00    9.660220e-02    -6.97674e-01    
1    5.000000e+00    -5.12333e-01    -4.06977e-01    
2    1.000000e+01    -1.12127e+00    -1.16279e-01    
3    1.500000e+01    -1.73020e+00    1.744186e-01    
4    2.000000e+01    -2.33914e+00    4.651163e-01    
5    2.500000e+01    -2.94807e+00    7.558140e-01    
6    3.000000e+01    -3.55701e+00    1.046512e+00    
7    3.500000e+01    -4.16594e+00    1.337209e+00    
8    4.000000e+01    -4.77488e+00    1.627907e+00    
9    4.500000e+01    -5.38381e+00    1.918605e+00    
10   5.000000e+01    -5.99275e+00    2.209302e+00    
11   5.500000e+01    -6.60168e+00    2.500000e+00    
12   6.000000e+01    -7.21062e+00    2.790698e+00    
13   6.500000e+01    -7.81955e+00    3.081395e+00    
14   7.000000e+01    -8.42849e+00    3.372093e+00    
15   7.500000e+01    -9.03742e+00    3.662791e+00    
16   8.000000e+01    -9.64636e+00    3.953488e+00    
17   8.500000e+01    -1.02553e+01    4.244186e+00    
18   9.000000e+01    -1.08642e+01    4.534884e+00    
19   9.500000e+01    -1.14732e+01    4.825581e+00    
20   1.000000e+02    -1.20821e+01    5.116279e+00    
\end{verbatim}
}
\caption{Tabular current outputs resulting from voltage sweep of VS1.}
\label{fig:spicecur}
\end{figure}
% }}}

% {{{ fig:spicecurplot
\begin{figure}

{\footnotesize
\begin{verbatim}
                      eece 311 laboratory project #1
           DC transfer characteristic Sun Sep 11 11:14:25  2011

Legend:  + = vs1#branch       
         * = vsens#branch     
--------------------------------------------------------------------------
 v-sweep    vs1#bran-2.00e+01        -1.00e+01         0.00e+00         1.00e+01
----------------------|----------------|----------------|----------------|
 0.000e+00  9.660e-02 .                .              * +                .
 5.000e+00 -5.123e-01 .                .               X.                .
 1.000e+01 -1.121e+00 .                .              +*.                .
 1.500e+01 -1.730e+00 .                .             +  *                .
 2.000e+01 -2.339e+00 .                .            +   *                .
 2.500e+01 -2.948e+00 .                .          +     .*               .
 3.000e+01 -3.557e+00 .                .         +      .*               .
 3.500e+01 -4.166e+00 .                .        +       . *              .
 4.000e+01 -4.775e+00 .                .       +        . *              .
 4.500e+01 -5.384e+00 .                .      +         .  *             .
 5.000e+01 -5.993e+00 .                .     +          .  *             .
 5.500e+01 -6.602e+00 .                .    +           .   *            .
 6.000e+01 -7.211e+00 .                .   +            .   *            .
 6.500e+01 -7.820e+00 .                .  +             .    *           .
 7.000e+01 -8.428e+00 .                . +              .    *           .
 7.500e+01 -9.037e+00 .                .+               .     *          .
 8.000e+01 -9.646e+00 .                +                .     *          .
 8.500e+01 -1.026e+01 .               +.                .      *         .
 9.000e+01 -1.086e+01 .              + .                .      *         .
 9.500e+01 -1.147e+01 .             +  .                .       *        .
 1.000e+02 -1.208e+01 .            +   .                .       *        .
----------------------|----------------|----------------|----------------|
 v-sweep    vs1#bran-2.00e+01        -1.00e+01         0.00e+00         1.00e+01
\end{verbatim}
}

\caption{Textual plot of current outputs resulting from voltage sweep of VS1.}
\label{fig:spicecurplot}
\end{figure}
% }}}

% {{{ fig:spicevolt
\begin{figure}
{\footnotesize
\begin{verbatim}
                         eece 311 laboratory project #1
                         DC transfer characteristic  Sun Sep 11 11:14:25  2011
--------------------------------------------------------------------------------
Index   v-sweep         v(3)            v(4)            v(5)            
--------------------------------------------------------------------------------
0    0.000000e+00    9.724494e+00    4.372093e+00    6.976744e+00    
1    5.000000e+00    1.249683e+01    5.883721e+00    9.069767e+00    
2    1.000000e+01    1.526917e+01    7.395349e+00    1.116279e+01    
3    1.500000e+01    1.804151e+01    8.906977e+00    1.325581e+01    
4    2.000000e+01    2.081385e+01    1.041860e+01    1.534884e+01    
5    2.500000e+01    2.358618e+01    1.193023e+01    1.744186e+01    
6    3.000000e+01    2.635852e+01    1.344186e+01    1.953488e+01    
7    3.500000e+01    2.913086e+01    1.495349e+01    2.162791e+01    
8    4.000000e+01    3.190320e+01    1.646512e+01    2.372093e+01    
9    4.500000e+01    3.467553e+01    1.797674e+01    2.581395e+01    
10   5.000000e+01    3.744787e+01    1.948837e+01    2.790698e+01    
11   5.500000e+01    4.022021e+01    2.100000e+01    3.000000e+01    
12   6.000000e+01    4.299255e+01    2.251163e+01    3.209302e+01    
13   6.500000e+01    4.576489e+01    2.402326e+01    3.418605e+01    
14   7.000000e+01    4.853722e+01    2.553488e+01    3.627907e+01    
15   7.500000e+01    5.130956e+01    2.704651e+01    3.837209e+01    
16   8.000000e+01    5.408190e+01    2.855814e+01    4.046512e+01    
17   8.500000e+01    5.685424e+01    3.006977e+01    4.255814e+01    
18   9.000000e+01    5.962658e+01    3.158140e+01    4.465116e+01    
19   9.500000e+01    6.239891e+01    3.309302e+01    4.674419e+01    
20   1.000000e+02    6.517125e+01    3.460465e+01    4.883721e+01    
\end{verbatim}
}
\caption{Tabular voltage outputs resulting from voltage sweep of VS1.}
\label{fig:spicevolt}
\end{figure}
% }}}

% {{{ fig:spicevoltplot
\begin{figure}
{\footnotesize
\begin{verbatim}
                      eece 311 laboratory project #1
           DC transfer characteristic Sun Sep 11 11:14:25  2011

Legend:  + = v(1)             
         * = v(3)             
         = = v(4)             
         $ = v(5)             
--------------------------------------------------------------------------
 v-sweep    v(1)     0.00e+00  2.00e+03  4.00e+03  6.00e+03  8.00e+03  1.00e+04 
----------------------|---------|---------|---------|---------|---------|
 0.000e+00  0.000e+00 + =$*     .         .         .         .         . 
 5.000e+00  5.000e+00 . X $ *   .         .         .         .         . 
 1.000e+01  1.000e+01 .  = X *  .         .         .         .         . 
 1.500e+01  1.500e+01 .   = $+ *.         .         .         .         . 
 2.000e+01  2.000e+01 .    = $  X         .         .         .         . 
 2.500e+01  2.500e+01 .    =  $ .*+       .         .         .         . 
 3.000e+01  3.000e+01 .     =  $.  * +    .         .         .         . 
 3.500e+01  3.500e+01 .      =  $   *  +  .         .         .         . 
 4.000e+01  4.000e+01 .       = .$   *    +         .         .         . 
 4.500e+01  4.500e+01 .       = . $    *  . +       .         .         . 
 5.000e+01  5.000e+01 .        =.  $    * .    +    .         .         . 
 5.500e+01  5.500e+01 .         =   $     *      +  .         .         . 
 6.000e+01  6.000e+01 .         .=    $   .*        +         .         . 
 6.500e+01  6.500e+01 .         . =    $  . *       . +       .         . 
 7.000e+01  7.000e+01 .         . =     $ .   *     .    +    .         . 
 7.500e+01  7.500e+01 .         .  =     $.    *    .      +  .         . 
 8.000e+01  8.000e+01 .         .   =     $      *  .         +         . 
 8.500e+01  8.500e+01 .         .    =    .$      * .         . +       . 
 9.000e+01  9.000e+01 .         .    =    . $      *.         .    +    . 
 9.500e+01  9.500e+01 .         .     =   .  $      .*        .      +  . 
 1.000e+02  1.000e+02 .         .      =  .   $     . *       .         + 
----------------------|---------|---------|---------|---------|---------|
 v-sweep    v(1)     0.00e+00  2.00e+03  4.00e+03  6.00e+03  8.00e+03  1.00e+04 
\end{verbatim}
}
\caption{Textual plot of voltage outputs resulting from voltage sweep of VS1.}
\label{fig:spicevoltplot}
\end{figure}
% }}}

The SPICE simulation can also be used to produce graphical plot output.
Using Ngspice the simulation must be re-configured to run in interactive
mode instead of batch mode so that the plot directives will work
(see the SPICE definition in Appendix \ref{sec:spicedef} for more information).
These plots can be displayed on the terminal or output to a file in various
format (jpg, png, eps, pdf) for inclusion in to a document.
Shown in Figure \ref{fig:spiceguiplot} is the graphical output
of this simulation when plotted using Gnuplot \cite{GNUPLOT}.

\begin{figure}[!hbtp]
\center
\includegraphics[scale=1.0]{spice/lab1-plot}
\caption{Graphical plot of SPICE simulation.}
\label{fig:spiceguiplot}
\end{figure}

\clearpage
% }}}

% {{{ Observations
\section{Observations}

The various ways of displaying data should corroborate the expected
and calculated values.
It can be seen by comparing the textual plot in Figure \ref{fig:spicevoltplot}
and the graphical plot in Figure \ref{fig:spiceguiplot} that the results
are the same other than the fact that the vertical axis is inverted.
The raw data can also be used to view the data in more detail than what
is available with the plots.
The textual plots are the least precise and have the largest amount of error 
compared to graphical plots.
% }}}

% {{{ Conclusion
\section{Conclusion}

This lab was a success in gaining experience with the various ways
to define and analyze circuits using SPICE.
A single SPICE circuit definition can be used to produce raw data values,
textual plots or full graphical plots suitable for inclusion in to a document.
% }}}

% {{{ References
\clearpage

\pagebreak
\renewcommand*{\refname}{\vspace{-8mm}}
\section{References}
\bibliographystyle{ieeetr}
\bibliography{../references}
% }}}

% {{{ Appendix
\appendix

% {{{ sec:spicedef

\section{SPICE definition}\label{sec:spicedef}
% spice/lab1.cir

{\footnotesize
\begin{verbatim}
EECE 311 Laboratory Project #1
* run using: ngspice -b lab1.cir
* http://www.ecst.csuchico.edu/~hma/Lab1.311.pdf

VS1 1 0 DC 100V
R1 1 2 3
R2 2 3 4
I1 2 3 DC 3.483

R3 1 6 10
VSENS 6 5 DC 0V
F1 3 4 VSENS 0.3

R4 3 0 12
R5 4 0 4
R6 4 5 2
I2 0 5 DC 2

* Uncomment the section for the mode your are using.
********> -b (batch mode) <********
** ngspice -b <this file>

* sweep VS1 from 0 volts to 100 volts in 5 volt increments
*.DC VS1 0V 100V 5V

*.PRINT DC I(VS1) I(VSENS)
*.PLOT DC I(VS1) I(VSENS)
*.PRINT DC I(R1) I(R2) I(R3) I(R4) I(R5) I(R6)
*.PRINT DC V(3) V(4) V(5)
*.PLOT DC V(1) V(3) V(4) V(5)
*.PROBE
*.END
******> interacitve mode <*********
** ngspice <this file>

.CONTROL

DC VS1 0V 100V 5V

* gui plot
*PLOT V(1) V(2) V(3) V(4) V(5)
GNUPLOT lab1-plot V(1) V(3) V(4) V(5)
* gnuplot -persist lab1-plot.plt
* outputs to lab1-plot.eps

.ENDC
.END
***********************************
* vim:syntax=spice

\end{verbatim}
}
% }}}

% }}}

\end{document}

% vim:foldmethod=marker

